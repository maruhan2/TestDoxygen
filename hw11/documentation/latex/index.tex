\begin{DoxyAuthor}{Author}
Kevin Mc\-Swiggen and Mars Park
\end{DoxyAuthor}
\hypertarget{index_intro_sec}{}\section{Introduction}\label{index_intro_sec}
If given a list of all words, it is possible to insert all those words in a hash table. Then, we can check if an abitrary word is a legitimate word by testing if it exists in the hash table.

The spell checker takes a step further and not only checks if the word exists in the hash but also suggests correct words. This spell checker does this by checking if there is a wrong letter in the word. When a word is not found in the dictionary, the spell checker will look up all variants that can be generated by changing one character. If the spell checker finds a match in the dictionary with one of these variants, it adds to the output line.

Our spell checker can be run using any of four choices of set implementation that inherit from the \hyperlink{class_abstract_set}{Abstract\-Set} abstract template class. \hyperlink{class_std_hash_set}{Std\-Hash\-Set} and \hyperlink{class_std_tree_set}{Std\-Tree\-Set} are wrapper classes for the S\-T\-L classes unordered\-\_\-set$<$\-T$>$ and set$<$\-T$>$ respectively. The \hyperlink{class_hash_set}{Hash\-Set} class is a custom implementation of a linear-\/probing hash set. When the key for a given element is taken, the insertion point advances until finding an unocupied index of the hash table. After the hash set becomes sufficiently full, it will resize the hash table and rehash the elements in the hash set to keep long clusters of keys from negatively impacting the performance of the hash set. The \hyperlink{class_tree_set}{Tree\-Set} class is a custom implementation of an A\-V\-L self-\/balancing binary search tree. When the subtrees of any node differ in height by more than 1, the tree is re-\/balanced by rotating through the node appropriately.\hypertarget{index_use_sec}{}\section{Usage}\label{index_use_sec}
The spell checking can be run using\-: \begin{DoxyVerb}./myspell [-d] [-h / -t/ -T / -H] dictionary
\end{DoxyVerb}


where
\begin{DoxyItemize}
\item -\/d prints information about the data structure used to represent the dictionary. \char`\"{}dictionary\char`\"{} is the file name of the dictionary.
\item -\/h Prints out\-: \char`\"{}n expansions, load factor f, c collisions, longest run l\char`\"{} where n, c, and l are integers, and f is a floating-\/point (double) number, printed in the default format.
\item -\/t Prints out a single line containing useful information about the structure of the tree, including its height.
\item -\/\-T Prints \char`\"{}\-No statistics available\char`\"{}
\item -\/\-H Prints \char`\"{}\-No statistics available\char`\"{}
\end{DoxyItemize}\hypertarget{index_change_sec}{}\section{Changes made to myspell program}\label{index_change_sec}
Myspell now keeps track of errors which have been previously seen and only prints out spelling corrections for novel errors.

Myspell is now implemented using \hyperlink{class_abstract_set}{Abstract\-Set}, and takes flages (-\/h, -\/\-H, -\/t, -\/\-T) to specify which type of set should be used to store the dictionary and set of previously-\/seen errors. The dictionary and error sets are constructed on the heap as the requested type.

Debugging information printed out by the -\/d flag now depends on which type of set is used to store the words.

performance\-\_\-page Peformance comparisons between the different sets can be found on their own page. 